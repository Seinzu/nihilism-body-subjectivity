\chapter{Conclusion}
\label{chap:conclusion}

As we have seen in the previous section there are several problems for Metzinger's analysis of Cotard's syndrome. Maintaining on one side, a view of the self that is largely dependent on the contribution of body and that loss of integrity of the self is consequently the result of loss of connection with the body while on the other an awareness that the empirical evidence seems to suggest that there is no such loss of connection does not seem to be possible. There are also problems with taking an approach to the syndrome that excludes a delusional element when the patient must overcome the evidence from her perception of her own body in order to declare herself dead.

There are also problems for Metzinger's analysis in terms of the causal force that he holds lead to Cotard's syndrome, depression. Based on the results of the largest surveys of the literature available \cite{berrios1995b}{enoch1991}, depression does not seem to be directly implicated in all cases, nor does the severity of depression vary concordantly with the severity of the symptoms. In terms of his secondary analysis (utilising the k\"{o}rper/leib distinction from Husserl) there are even more problems, bundled within the leib are many world-facing abilities which enable much of the complex behaviour that humans are capable of (see section \ref{metzinger_self} for more on this).

To hold together the complex abilities needed to function in the complex environment and maintain the complex internal situation that human beings have requires Metzinger to consider splitting his equivalent of pre- reflective self-awareness, stripping it of subjectivity and leaving perspectival orientation on the body. There are two problems with this, (1) there is little justification for the theory that we might lose one type of functional content that is focussed on the self while maintaining another that shares the same focus and (2) other conditions such as OBE seem to involve a perspectival centre that is focussed on the self rather than on the body, suggesting that the two might be intrinsically linked and that separation of the sort suggested by Metzinger may be simply impossible. Of course, neither of these problems is fatal to Metzinger's position, however in concert with the problems with his analysis generally these problems become more severe. It seems that Cotard's syndrome functions for Metzinger as a proof that subjectivity and perspectivalness can be separated and if his analysis does not hold up (and it seems that it does not) then there are problems for this separation and/or his general theory equating the self with the body.

Although Metzinger's analysis of Cotard's syndrome is not successful there is still much to be gained from attention to his theory in general and from the attention that he has paid to specific disorders like Cotard's syndrome. There is a great deal of importance in investigating Cotard's syndrome specifically. Statements such as“I am dead” and “I do not exist” made with conviction by a person who is incorrigible regarding their experience ofthe mental content leading to the utterance of them provide a stern test for any model of the self.As well as weeding out weaker candidate theories of self a full explanation of the condition will enable us to refine our conception of self, as Metzinger puts it, a proper treatment of the syndrome must be incorporated into “any good future philosophical theory of mind” \cite[p. 455]{metzinger2003}.

The flexibility of the general approach taken by Metzinger means that even with his primary and secondary analyses of Cotard's syndrome in difficulty we are still able to find ways to incorporate explanation of it into Metzinger's framework and I have outlined some of the possibilities in this domain.It is clear that further work needs to be done on fully understanding the symptomotology of Cotardian patients before we can ascertain the exact cause of the syndrome even on a broad functional level.

Finally, in the course of this paper I have brought into the foreground several of the writers in the phenomenological tradition that Metzinger has relied upon (most notably Husserl and Sartre). In spite of his famous disagreements with contemporary phenomenology, Metzinger does a very good job of respecting the work of the phenomenologists and of bringing together the analytic and phenomenological tradition, a union that may be very important in getting to grips with the extreme difficulties posed by unifying brain and mind.