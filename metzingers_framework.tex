\chapter{Metzinger's Framework}
\label{chap:framework}

Before discussing Metzinger's analysis of Cotard's syndrome it is worth investigating those elements of Metzinger's framework that have the most impact on the analysis of Cotard's syndrome, the Phenomenal Model of the Intentionality Relation (PMIR) and the Phenomenal Self-Model (PSM). The other major element of Metzinger's framework is the description of a series of constraints that he places on conscious experience. Aside from three (transparency, adaptivity and global availability) which play an explicit role in the explanation of Cotard's, these criteria need not be explained in order to grasp the more directly related elements mentioned above, so I will not investigate all of the constraints here.

One of the most important aspects of Metzinger's framework is the adoption of multiple-levels of explanation for the exploration of conscious experience, each of the properties defining conscious experience are investigated on each of several levels and a methodology which is focused on attempting to answer the difficult questions posed by consciousness by utilising these levels, in unison or separately, is employed. In general, the level on which the discussion here will be focused is the phenomenal (and to a lesser extent the functional) because Metzinger's analysis of Cotard's syndrome deals mostly with changes that occur on this level.
The PSM is a mental model constructed from "your current bodily sensations, your present emotional situation, plus all the contents of your phenomenally experience cognitive processing" \cite[p. 299]{metzinger2003}. The subject is not aware that the PSM, which constitutes the experience of having a self, is a model, this state of affairs being referred to by Metzinger as the "transparency" of a piece of representational content.

The PMIR is another "conscious mental model"\footnote{The notion of the mental model as Metzinger uses it is drawn from Johnson-Laird's work on mental modelling \cite[p. xi; 38]{metzinger2003}. Metzinger treats the contents of phenomenal experience as taking the form of a model of the external world, mental models may be "embedded in each other" (39 p. 426) as the phenomenal self model is embedded within the phenomenal model of the intentionality relation (for instance).} in which "the content is an ongoing \ldots subject-object relation" \cite[p. 23]{metzinger2005}. It is characterised by three elements, a representation of the self (normally the PSM), a representation of an object, which may be external or internal, and the relationship between these two things represented by Metzinger, with shades of Brentano, as an intentional arrow. As mentioned previously we might attend to an internal object (this will be important in section 3 of the paper), even to the extent of the object being an introspective self-referential state along the lines of "the feeling I am having on looking at this book"\footnote{As opposed to "I am someone who is currently visually attending to the color [sic] of the book in my hands" \cite[p. 23]{metzinger2005} which is an order less introspective than my example.}.
Earlier I mentioned that the model nature of the PSM is transparent. As this terminology will have some importance in understanding Metzinger's analysis it is worth a little exposition on transparency as a property of mental content. Representational content can be either transparent or opaque (although these are not exclusively binary states, but may also be a continuum – for instance in self-presentational contexts \cite[p. 332]{metzinger2003}). If representational content is opaque the fact that it is produced by a representation is available to the subject, with the opposite condition holding for transparent content.

The two other constraints that are specifically invoked in the explanation of Cotard's syndrome offered by Metzinger are global availability and adaptivity. I will not approach these exhaustively here but rather offer a brief outline and return to these when they can be placed in their direct context during section \ref{metzinger_self}. Global availability is most relevant to Cotard's syndrome on the functional level. At this level there are some similarities to Baars Global Workspace Theory \cite{baars1988} although Metzinger is keen to stress that such similarities are only general rather than in terms of the specifics of architecture or neurobiology \cite[p. 120]{metzinger2003}. Conscious information is globally available information, by which it is meant that the information fits into a world-model that makes it available to diverse information processing modules across the brain \cite[p. 121]{metzinger2003}. This means that information can be available to the system in terms of appearing in subsystems but not attain global availability and thus not be playing a role, or appearing at all, in the subject's phenomenal reality.

Adaptivity is simply the idea that for conscious experience to be present it must have a teleofunctional role in a species' evolution and work to enhance the survival possibilities of individuals. If this is the case then we may be able to gain some insight into the way specific phenomenal experiences or experience classes operate by trying to understand them in terms of how they aid the organism in maintaining itself and improving its survival prospects.

This constraint is relevant to Metzinger's theory of self and by extension his analysis of Cotard's syndrome, so I shall return to consider this topic in more detail as part of section \ref{metzinger_self_adaptivity}.

\section{Metzinger's analysis of Cotard's syndrome}

Metzinger splits cases of Cotard's syndrome into two classes on the basis of severity. In the most severe cases the patient may claim to be non- existent, whereas in less severe cases the patient claims to be dead or makes other nihilistic claims regarding her body (that parts of it are missing or have been replaced or that it is rotting).

In the more severe class of cases we can find thoroughgoing nihilistic claims such as a denial of the existence of the world and the self. In some cases, such as the patient of Anderson cited by Enoch and Trethowan\footnote{Also cited by Metzinger on p459.} \cite[p. 173]{enoch1991} who uses the word "it" in referring to herself rather than the typical usage of the first person pronoun "I"\footnote{The example given by Enoch and Trethowan \cite{enoch1991} is the sentence "...wrap it up and throw ‘it' in the dustbin". As this statement is rather ambiguous without greater context (for example it might be a reference to the patient's body or the self as a whole), it is unfortunate that the reference given by Enoch and Trethowen \cite[p. 136]{tissot1921delire} does not appear to contain the sentence referred to or contextualising information regarding it.}. Metzinger takes this to show that the patient has lost their representation of "self as subject" \cite[p. 461]{metzinger2003}; that the perspectivally centred self is no longer treated as a subject but rather an object. We might view this in light of Sartre's \cite{sartre1956}\cite{sartre1972} observation that the "I" as used in Descarte's cogito \cite{descartes1998} is a reflection upon a self that exists in experience without being experienced as an object. The Cartesian "I" itself can exist only as an object of experience. Placed in this context we can see that Metzinger is making the claim that the subjective self is completely missing from the severe Cotard's syndrome patient's reality. I will return to Metzinger's discussion of this and the equivalent conclusion he makes for the less severe form of Cotard's syndrome later in this paper.

Cotard's syndrome is generally considered to be a "continuum" syndrome \cite[p. 173]{enoch1991}\cite[p. 148]{young1995} and Metzinger appears to agree with this consideration. For Metzinger the cause of the syndrome seems most likely to be what he refers to as the "‘emotional disembodiment' conjecture" \cite[p. 458]{metzinger2003} with the difference in the severity of the syndrome corresponding to the difference in degree of emotional disembodiment.

What exactly does Metzinger mean by "emotional disembodiment" and what reasons does he give for concluding that it is the key factor distinguishing Cotard's syndrome\footnote{Metzinger observes that "there are a number of other phenomenological state classes in which a person may experience herself as bodiless or disembodied" \cite[p. 457]{metzinger2003}) and we must note that in the cases mentioned by Metzinger ("out-of-body experiences...and lucid dreams", to which we can probably add some dissociative states) the unique claims made by Cotard patients are not made. This suggests that there is a unique mechanism at work distinguishing the Cotard phenomenology (or at least the reporting behaviour of the Cotard patient) from the other disembodied phenomenological state classes. I will discuss these states a little more later in this paper.}? Before I investigate Metzinger's answer to this question, it would be useful to lay out some of the potential ways that we could understand the word "embodiment" in the emotional context.

\section{Embodiment}

Embodiment is a word often used in the context of the enactive approach to cognitive science. As used in this sense embodiment does not have a lot to do with the role it plays in the emotional disembodiment conjecture. O'Regan and No\"{e} \cite{oregan2001, noe2006} discuss the theory that our visual experience is constituted by knowledge of sensorimotor contingencies. Facts about our bodies and their relationship to the environment determine the nature of our experience. Sensorimotor contingencies are either part of the sensorimotor system or facts about the environment, hence O'Regan and No\"{e}'s position is an externalist one.

Clark and Chalmers \cite{clark2002} develop another externalist position regarding embodiment when they make the claim that the vehicle of mental content\footnote{They present the argument that for a sufferer of Alzheimer's a notebook with a list of beliefs is functionally equivalent to the storage of beliefs in the head by a person who does not suffer from Alzheimer's.} can often be part of the environment. Both of the preceding theories briefly outlined can be seen as treating the body (and by extension its environment) as playing an important role in cognition. Metzinger opposes this when he makes statements like: "...every form of phenomenal content...supervenes on internal and contemporaneous functional properties of the human brain" \cite[p. 415]{metzinger2003}.

It is clear that to understand embodiment as Metzinger uses it we must make the counter-intuitive move of confining our attention to the brain and ignoring the body. Our embodiment is, on Metzinger's account, a form of representational content based on "nonconceptual knowledge15 about the presence and current state of one's own body" \cite[p. 287]{metzinger2003}. When Metzinger refers to embodiment we should understand that he is referring to an internal (extant within the brain) model of the body and the sensations that are transmitted from the body to the brain. Damasio puts forward the idea that "the organism, as represented inside it's own brain, is a likely biological forerunner for what eventually becomes the elusive sense of self" \cite[p. 22]{damasio2000} and Metzinger's use of the term embodiment can be seen as informed by this. I will come back to exactly how Metzinger deals with the notion of disembodiment in the next section.

Metzinger follows Ahlheid \cite{ahlheid1968} in drawing the conclusion that "the Cotard's patient has a bodily self-model as a K\"{o}rper, but not as a leib" \cite[p. 457]{metzinger2003}. Gallagher \cite{gallagher2005} points out that these two ways of considering the body as presented in phenomenal experience is present in Husserl's Ideas \cite{husserl1931} but notes that the way Metzinger treats k\"{o}rper diverges from Husserl's treatment. The distinction Husserl makes is present in the works of many others in the phenomenological tradition, appearing in Merleau-Ponty \cite{merleauponty1962} and by way of Merleau-Ponty coming to be introduced into cognitive science by Varela, Thompson and Rosch who present a double sense of embodiment in the following way "we see our bodies both as physical structures and as lived experiential structures" \cite[p. 1]{varela1991}. The leib can be understood as equivalent to the lived experiential structure (and as Gallagher notes this is how Metzinger correctly treats it).

Of course the k\"{o}rper fulfils the other sense of embodiment, embodiment as seeing our body in the form of a physical structure, as from outside. Gallagher picks up on Metzinger's treatment of the k\"{o}rper "as inanimate object" (1 p. 457) and also makes the claim that "Metzinger characterises the k\"{o}̈rper...specifically as the dead corpse"\footnote{It should be noted that the German word for corpse is "die leiche" which maintains linguistic, etymological links to leib.} \cite[p. 3]{gallagher2005} rather than "as it is studied by science, or...by someone who...may experience their own body...'from the outside'\footnote{An example of this might be viewing oneself in a mirror or in film footage.}"\cite[p. 3]{gallagher2005}. However, it is not clear where Metzinger explicitly makes this mistake (I will come back to an implicit case of this in the next section). In fact, elsewhere in his book, Metzinger seems to fully respect the distinction as Gallagher presents it "it is particularly interesting to note how the self-model simultaneously treats the target system ‘as an object' (by using proprioceptive feedback...) and ‘as a subject' (e.g. by emulating its own cognitive processing in a way that makes it available for conscious access)" \cite[p. 301]{metzinger2003}.

Metzinger has something of a mixed relationship with phenomenology. Although he claims that his use of the k\"{o}rper versus leib distinction stems from a "conceptual distinction, which is available in the German language" his presentation of the words is not in keeping with standard German usage\footnote{Leiche would be a more obvious choice in the places where he employs k\"{o}rper.} or if it is (there is some considerable ambiguity in his usage) he is using the word in a very similar way to the way in which Husserl, and much of the rest of the phenomenological tradition, uses it. It would seem somewhat of a stretch to claim that Metzinger as a native German speaker is making such a simple linguistic error so I am moved to conclude that he has been influenced here by the use of these terms by the phenomenologists (specifically Husserl). It is also the case that his framework as a whole respects the phenomenological level of description and in many places has directly benefited from work in the phenomenological tradition. His project can be seen, at least in some aspects, as an admirable attempt to find a much-needed rapprochement between the Continental and Anglo-American (particularly within the Cognitive Sciences) philosophical traditions. This openness to phenomenology seems strange in the light of his reported comments at Elsinore dismissing "phenomenology as a discredited research programme" \cite[2,385]{gallagher2008,metzinger1995faster}.

One of the ways we can view Metzinger as diverging from the phenomenological orthodoxy is the disagreement about the degree to which the body is central to understanding embodiment that he engages in with Husserl, Merleau-Ponty and others\footnote{Heidegger does not deal much with the body and so does not explicitly disagree with Metzinger on this topic. Metzinger's other main diversion from phenomenological orthodoxy is in embracing mental representation, which is where he disagrees fundamentally with Heidegger.}.

This disagreement can be expressed more starkly by contrasting him with the views of embodiment that I described at the very beginning of this section. Metzinger's view of embodiment is not world-involving like the vehicle externalism suggested by Clark and Chalmers, it is not even body-involving like O'Regan and No\"{e}'s\footnote{O'Regan and No\"{e}'s position is also world-involving in that some of the sensorimotor contingencies are present in the object itself (e.g. the example given of properties of a line defining our reaction to it).} theory of sensorimotor contingencies. Merleau-Ponty treats human cognition as a form of skilled knowledge highly dependent on the body and devoid of mental representation in an information processing sense (relying on Heidegger's treatment of the issue). The body (and body-in-the-world) is present in Metzinger but it is not present directly in the way that it is in phenomenological analyses, instead it comes to be known in a mediate way through proprioceptive representations and emotional states.

\section{Metzinger and emotional disembodiment}
\label{metzinger_emotional_disembodiment}

For Metzinger "Cotard's syndrome may be analysed as a combination of loss of a whole layer of non-conceptual transparent content and a corresponding appearance of new, quasi-conceptual and opaque content in the patient's PSM" \cite[p. 458]{metzinger2003}, specifically the content lost is the emotional representation of the body leaving only an opaque perception of the body as inanimate. Further, "the fact that autonomous self- regulation...is going on ...is no longer part of the [Cotard's syndrome] patient's reality" \cite[p. 459]{metzinger2003} accounting for the false belief of death commonly found in Cotard's patients. Working from the assumption that the causal factor determining the development of nihilistic delusions is a substitution of a self-as-k\"{o}rper for the self-as-leib Metzinger outlines the following argument to get to the loss of emotional embodiment:

\begin{enumerate}
    \item The leib becoming "a representation of something inanimate" is a consequence of a loss of the "logic of survival" \cite[p. 457]{metzinger2003}
    \item "The logic of survival is...made globally available by conscious emotions" \cite[p. 457]{metzinger2003}
    \item Thus, impairment of the (conscious) emotional self- model is the cause of Cotard's syndrome.
\end{enumerate}

The move that Metzinger makes in step 1 (supported by a link he makes between the "inner logic of life" and the "logic of survival") does seem to back up the observation that Gallagher made about Metzinger confusing k\"{o}rper in the Husserlian sense and the common-usage German word k\"{o}rper. Metzinger may be making a claim such as: k\"{o}rper without leib is an inanimate, dead body (this is backed up by the linguistic origin of corpse in German as explained in a previous footnote). To validate such a claim we need to provide a reason why in non-empathic interactions with other living beings they (the creatures) are not considered as dead or inanimate. For example, it would seem acceptable to not attribute a leib to an ant (or a single-cell organism) but we would not wish to claim that these items have only a dead or inanimate body.

We can see therefore that the use Metzinger makes of the word "inanimate" in Anglicising k\"{o}rper is a somewhat misleading choice\footnote{There is however some support for this decision in Smith, "[leib] refers to an animate, living body...[k\"{o}rper] refers to any ‘material body'" \cite[p. 220]{smith2003}. So, it is possible that a k\"{o}rper could be an inanimate body and we might read Smith as suggesting that a k\"{o}rper on its own, lacking the leib with its provision of animation and vitality, must be inanimate and dead.}. We should consider that the view of the body from outside respects the fact that the body is living.

Zahavi notes that "bodies of others differ radically from inanimate objects, and...our perception of these minded bodies is unlike our ordinary perception of objects" \cite[p. 155]{zahavi2005}. When we come to be aware of others we are aware of them first as a k\"{ö}rper (a living, but not lived, body) and by empathy we might come to have an awareness of them as a leib (although not in the precise way that we are aware of the same in our selves). We can see this at work within Metzinger's philosophy when he emphasises the ability we have to simulate or emulate other people \cite[pp. 300-301]{zahavi2005}\footnote{He specifies this in terms of information processing systems for which we could substitute something like "cognising agents".}. He also approvingly cites Lipps as analysing "the representational content of empathy...as feeling yourself in an object" which seems to suggest that when we emphasise we are applying our own experience of leib (with suitable variable adjustment) to fill in for the leib of the other.

From this discussion it is clear that it is a consequence for Metzinger's explanation of Cotard's syndrome that the patient would suffer from an inability to effectively empathise with others. As we can see, the loss of leib as a component in the PMIR should be expected to have an impact on the social behaviour of the patient. Further, we would expect to see more general problems for the sufferer of such a loss. Smith observes that for Husserl "my body is not only the locus of ‘emotional' feelings and ‘bodily' sensations, it is a sensitive body: when I am touched I typically feel it"\footnote{Smith uses "body" when referring to the leib (as previously noted he uses "material body" to refer to the k\"{o}rper).} \cite[p. 221]{smith2003}, we should therefore expect the sufferer to exhibit anaesthesia and/or insensitivity to touch.

The leib also plays a key role in orienting the subject in the world and producing effortless motion of the material thing through the world \cite[p. 221]{smith2003}. We might expect therefore to see difficulties for patients in distinguishing their location, the boundary between themselves and their environment, and in controlling their bodily motions.

Although there is much in this discussion of the phenomenal literature, a literature made much use of in his analysis, Metzinger also restates his diagnosis in terms more directly couched in his explanatory framework. The Cotard's patient has a PSM that "no longer contains the information that actual elementary bioregulation is going on" and the sufferer loses "any subjective representation of the current degree of satisfaction of the adaptivity constraint" \cite[p. 457]{metzinger2003}. In this reference to the adaptivity constraint we have the emotional disembodiment that Metzinger considers the chief deficit evident in Cotard's patients.

For Metzinger the chief role of emotions is as "[arbiters of] the adaptivity of a certain situation...or [as] a form of self- representation...portraying bioregulatory aspects of the organism's own bodily state" \cite[p. 199]{metzinger2003}. It is in this second sense that the failure of emotional responsiveness causes problems for Cotardian patients. With the posited failure of emotional embodiment the patient has the experience of having a body from which they get no feedback regarding its internal state and so believes their body to be dead. This is sufficient for the less severe form of Cotard's syndrome (refer back to my division of Cotard's patients by severity in the second section of this paper). On top of this the severe form is indicated by "a persistent false belief" \cite[p. 459]{metzinger2003} that the subject does not exist at all. This false belief comes about for one of three potential reasons that Metzinger puts forward:
\begin{enumerate}
    \item The patient's "self model bec[o]me[s] fully opaque" \cite[p. 460]{metzinger2003} and thus, contravening the transparency constraint on conscious content, disappears from phenomenal experience.
    \item "There is no longer a PSM in existence"\cite[p. 460]{metzinger2003} at all.
    \item The PSM remains transparent but rather than being a subject model it becomes an object model, thus causing the PMIR to become "a model of...an object- object relation" \cite[p. 460]{metzinger2003} rather than modelling a subject-object relation.
\end{enumerate}

Metzinger dismisses the second theory as empirically implausible because of the ability that the Cotard patient has at interacting with the world and sensorimotor coordination. This highlights the idiosyncrasy of Metzinger's interpretation of the effects of the loss of the leib, as these are two of the duties assigned by Husserl to the leib (per Smith), the way in which the alternative to this is distinguished will be discussed in some detail during section \ref{metzinger_self_sartre}.

The first hypothesis is considered by Metzinger to be the cause of the spiritual experiences of an absent-self noted in Buddhism and several other Eastern religions (along with some experiences caused by the influence of entheogenic substances). He does not seem to consider it a plausible explanation for Cotard's syndrome, so I will not consider it further save to note that the key difference between the first and third explanation is that awareness of the self model's empty content is available in the first explanation and not in the third. The difference here does not seem to be on the level of the actual pre-reflexive experience but rather with regards to some higher-level conceptual conclusion drawn on the experience that the person undergoing the experience has.This seems as if it will explain the difference between the serene individual who has achieved ego-loss through spiritual revelation and the deeply troubled sufferer of Cotard's syndrome. In terms of the discussion in the next subsection (2.4.) there is no difference between these two cases.
With the other two hypotheses being discarded the third is evidently considered to be the best candidate explanation of Cotard's syndrome for Metzinger.

\section{Metzinger's conceptualisation of the self: a re- examination}
\label{metzinger_self}

Before progressing to the next section of this paper it is important that some clarification is made of Metzinger's view of the self. In the foregoing there may be some suspicion that I have been presenting Metzinger's position in a way that he would not assent to. After all, as I noted earlier Metzinger has a history of being very critical of the phenomenological tradition and yet I have presented much of my discussion of his treatment of Cotard's syndrome in terms of a phenomenological distinction (the k\"{o}rper/leib distinction). For that part of the paper I can only rely upon Metzinger's own use of that very same distinction and the fact that Ahlheid, whom Metzinger prominently credits as being the originator of the aforementioned treatment, is very clearly speaking in terms of Husserl influenced phenomenology. As to the other mode of presentation (that which is framed in his own terminology) made use of by Metzinger, I intend to show that (at least on one level of explanation – the phenomenological) he does not for the most part diverge strongly\footnote{I will cover the one way in which he might be seen to diverge from the phenomenological tradition in section \ref{metzinger_self_sartre}. In section 3.2. I will clarify the effects that the divergence has on Metzinger's analysis of Cotard's syndrome.} from the phenomenological tradition and that I have not helped myself to too much in my discussion of his approach to Cotard's syndrome. Beyond such important formal considerations we can expect to gain a deeper insight into Metzinger's theory and how the explanation of Cotard's syndrome fits into it, which will make this slight detour valuable on its own account.

I have, in the course of the preceding examination, drawn from the phenomenological literature many claims that should hold if Metzinger's analysis of Cotard's syndrome is correct\footnote{These can be found summarised in the next section (3.2.) or during the course of the exposition in the previous sections.}. In order to show the validity of these observations I will now proceed to outline the relevant parts of Metzinger's view of the self\footnote{It is important to make clear that the present section is not expected to hold Metzinger's theory up to any epistemic standards or to make any claims regarding the plausibility of his view of the self as opposed to any other; the aim is purely (as stated) that the validity of the comparison made with phenomenological sources is supported and that a level of understanding of the theory is available for the discussion later in the paper.} and contrast this where appropriate with the conclusions I have drawn from the phenomenological tradition (some of the discussion will drift below the phenomenal level and hence will be inappropriate for direct comparison).

\subsection{Adaptivity and the roots of self}
\label{metzinger_self_adaptivity}

Undoubtedly the most sensible starting point for the discussion of this view of the self is in its origination. The question of why a conscious self has been developed, at great cost as Metzinger notes \cite[p. 347]{metzinger2003}, in humans will motivate many of the claims that are made about the teleofunctional role of the self in the cognitive economy. The feeling of having a self, of being someone is, for Metzinger, "...rooted in elementary bioregulatory processes" \cite[p. 345]{metzinger2003} which maintain within the organism as great a degree of invariance in internal state as possible. This invariance creates a self-similarity that enables a self-conscious entity to experience itself as a consistent being across time. At this point it is worth observing that such processes (those that maximise invariance) are present in entities that we would not wish to credit with self-awareness. In such cases, these processes must be conducted on a subconscious level (for Metzinger the difference is one of global availability for cognition); this of course still leaves outstanding the question of the utility of the conscious experience of these processes (Metzinger does not consider conscious experience to be epiphenomenal, so there must be some utility to the possession of phenomenal experience).

Self-consciousness provides the ability for regulation to come under endogenous control from the organism, which, in Metzinger's terms, provides a greater degree of flexibility\footnote{Metzinger states that "...the step from nonconscious to conscious representation is the step from rigidity to flexibility" \cite[p. 348]{metzinger2003}. This is precisely the ability to break free from our genetic "programming" and select a course of action that serves a longer term goal.}. It also provides the organism with the ability to place its own current state accurately within a broader, external context provided by the world that it is located in. However, no definitive, firm line is drawn to show the contribution made by conscious availability of the self-model and those capabilities that slip out of reach of those organisms that lack a conscious self-model.

It is worth observing that Metzinger's treatment of the self bears a great debt, frequently acknowledged, to the work of Damasio (particularly his monograph The Feeling of What Happens). Paralleling Metzinger's theory of the origin of the self and subjectivity, Damasio notes that the self\footnote{The self generally referred to by Damasio is called the "core self" by Metzinger and Damasio.} "...must possess a remarkable degree of structural invariance so that it can dispense continuity of reference...continuity of reference is what the self \emph{needs}\footnote{The emphasis here is mine} to offer" \cite[p. 135]{damasio2000}. Having an entity in the world that we can reliably perceive as belonging to us provides us with the ability to "relate to various objects in space...[and] consistently react emotionally in a certain way to certain situations" \cite[p. 135]{damasio2000}. To a varying degree these abilities, or functionally equivalent ones, are available to even the simplest of organisms. As a complex nervous system and eventually conscious experience are brought into play the organism is able to control a more complex internal situation as well as manage more complex interactions with the environment that it is situated within \cite[p. 139]{damasio2000}. This seems to be equivalent to the movement towards flexibility cited by Metzinger as the advantage bestowed upon conscious organisms.

We cut to the centre of Metzinger's treatment of the self when we understand that although this "remarkable degree of structural invariance" provides the appearance of a consistent entity, the self as we all experience it, there is no such consistent entity. Perhaps eventually there will be found an area (or a spatially dispersed pattern) of the brain that seems to possess the necessary invariance to support the self\footnote{Damasio acknowledges this possibility in the following passage "Our search for a biological substrate for the self must identify structures capable of providing [the necessary] stability" [ibid. p. 135]. Metzinger also makes just such an acknowledgement with his statement that "the place where the soul is most intimately linked to the body...is not Descartes's pineal gland, but rather the upper brainstem and the hypothalamus." \cite[p. 292]{metzinger2003}} but we will still have not found the self as construed by common intuition. Viewing the self as a transcendent ego or soul is to mistake a constantly changing (even if only very gradually) model generated as part of a process of systemic regulation for an entity that stands apart from the organism and the process contained by it that generates this model.

Damasio's view of the self has further relevance specific to Metzinger's analysis of Cotard's syndrome. In Metzinger's analysis the element of phenomenal experience that is missing in the case of the Cotard's syndrome patient is the transparent PSM (it's not truly missing but merely becomes opaque, I will return to this later in this section). Metzinger states that the "concept of a transparent PSM bears at least some similarity to Antonio Damasio's notion of the core self" \cite[p. 340]{metzinger2003}. We should therefore expect to see some consequential effects on the abilities and knowledge embodied by the core self as a result of the loss of pre-reflective self-awareness in the Cotard's syndrome patient (more on pre-reflective self-awareness in section \ref{metzinger_self_sartre}).

\subsection{Comparison of self in Metzinger and Husserl}
\label{metzinger_self_husserl}

With Metzinger's thoughts on the origin and utility of the self described in the preceding section, we should now turn to discuss what similarities can be found between this account and Husserl's account of the leib which is the treatment of self most pertinent to this discussion of Metzinger's analysis of Cotard's syndrome within the phenomenological tradition.

As the element of Husserl's theory in question deals with two modes of bodily awareness we are already close to a similar position to Metzinger's which emphasises the contribution of bodily awareness to a sense of self. There is a difference between later Husserl\footnote{Early Husserl \cite{husserl2001shorter} took a view of the self that was similar to the view that Metzinger takes (at least insofar as it excludes a transcendental ego) whereas later Husserl was explicitly egological. Further reading on this topic can be found in the fifth investigation in Husserl's logical investigations or in Gurwitsch \cite[pp. 287-300]{gurwitsch1966nonegological}.} and Metzinger as later Husserl believes in the existence of a transcendental ego whereas Metzinger does not (recall "no such things as selves exist in the world"). However, when discussing the body and the k\"{o}rper/leib distinction this difference becomes moot. As Husserl puts it "in my waking consciousness I find myself...without ever being able to alter the fact...in relation to the world" \cite[p. 53]{husserl1931}, there is never an instant when the self is not in-the-world or embodied so the difference in the two views is largely irrelevant for this discussion.

I have explained the contribution made by the leib elsewhere in this paper, what remains is to investigate which of these elements have their analogue in Metzinger's discussion of the bodily self. I have covered the similarity between Metzinger and Husserl's views of empathy previously in section \ref{metzinger_emotional_disembodiment} so will not need to investigate it further here. I will also deal with some issues to do with self-location and perspectivalness in section \ref{metzinger_self_sartre}. The main elements I want to compare at this point are mineness and the self-world boundary which we will find have very similar treatments.

The leib is useful in distinguishing that which belongs to me from that which is other. That which comes under my willed control belongs to me. This seems to be a neat parallel of the property of mineness as invoked by Metzinger \cite[p. 302]{metzinger2003}. The order of primacy comes differently for Metzinger than for Husserl. For Husserl, the self experiences mineness and the self-world border is constituted. For Metzinger the order is the opposite, first there is a self-world boundary and then a self-model comes into being \cite[p. 313]{metzinger2003}. Self is necessary, on Husserl's view, for the generation of a self-world boundary but not sufficient (we must have a perception of body and ownership of it in order to draw the boundary). On Metzinger's view the self-world boundary is sufficient and necessary \cite[pp. 307, 313]{metzinger2003} to generate a self-model. For both then, the existence of a self-world boundary implies the existence of a self.

\subsection{Metzinger, transparency and the Sartrean pre-reflexive self}
\label{metzinger_self_sartre}

Earlier in the paper, amongst a selection of Metzinger's constraints on conscious experience, I briefly discussed transparency. Transparency plays an interesting role both in the gestalt of Metzinger's view of the self but also in his specific treatment of the self as affected by Cotard's syndrome. All phenomenal content is, for Metzinger, representational in nature. What makes an element of phenomenal content transparent is that the subject experiencing the content is not aware of the representational nature of the content. 

In Cotard's syndrome Metzinger predicts that "a human being's self model becomes fully opaque" \cite[p. 460]{metzinger2003} which causes the person thus afflicted to experience her own nonexistence. We can contrast this with a statement that Metzinger makes earlier in his book when he says that "cognitive self-modelling generally is opaque, although it \emph{always} possesses a transparent component"\footnote{The emphasis here is mine.}. Before tackling the apparent contradiction between these two statements it will be useful to make clear a link between the phenomenological sources I have brought into play and this specific issue.

Earlier I mentioned that Sartre, when talking about pre-reflexive self-awareness and the Cartesian cogito, distinguishes between the self as it is given as a subject and the self as it enters reflective thought as an object. In some regards this is much the same as the treatment given by Metzinger of the self. For Sartre, the "I" as referred to in the cogito is a representational element, when we refer to our self in a reflective thought we are constructing a representation of our self and using it to stand in a relation to some condition or property (that of being in this case). This representational content for the most part still remains transparent in the sense that the subject feels themselves to be identical to the representational object "I". The pre-reflexive experience of self is always present and always transparent (for Sartre it is not representational at all, so it is not that it is representational that is transparent) as we cannot become aware of it in self-directed thought without the awareness being of an objective representation. This neatly parallels Metzinger's assertion about there being a transparent element ever present in the self-model.

What exactly is this transparent element and is it equivalent to the Sartrean pre-reflective self? Metzinger says that "there are aspects...of ourselves...the concreteness and immediacy of bodily presence and some elementary emotions, which we never experience as mental" \cite[p. 332]{metzinger2003}, "mental" in this sense meaning an object of thought, something which we can recognise as a misrepresentation of reality, in essence content that may become opaque. This is not enough to bring us to a Sartrean view but this later comment may bridge the gap: "[in some situations] we do not distance us from ourselves by generating higher-order self-representational content", at these times "our PSM is devoid of any opaque content" \cite[p. 333]{metzinger2003}. We become absorbed in an action or in our situation generally and there remains no content of an openly representational nature within our conscious experience. Most important though is the fact that we are set up for a distinction between two selves "us" and "ourselves" seeming to refer to some nonconceptual perspective on the world as well as to a conceptual perspective that we can reach "by generating higher-order self-representational content". Finally, very strong support for the connection between Metzinger and Sartre comes in this statement: "cognitive self-reference always takes place against the background of transparent, preconceptual self- modelling". This preconceptual self-modelling is precisely the cause of the phenomenal experience of pre-reflexive self- awareness.

Metzinger teases apart the pre-reflexive self further than this however. It is the pre-reflexive self that is the constant centre of our experience in much the same way that the leib is in Husserlian phenomenology. Also like the leib, it is the source of the mineness that all of our experience has; a sense that we own our perceptions, both external and internal. Metzinger pulls apart these two elements and says that we can have "systems that functionally operate under an egocentric frame of reference...while at the same time phenomenally operating under a nemocentric reality model" \cite[p. 336]{metzinger2003}\footnote{The emphasis here is as in the original. A further note regarding terminology might be necessary here, nemocentric is a representation "that is centred on nobody" \cite[p. 336 (footnote)]{metzinger2003}, due to Grush \cite{grush2000}.}. Although this is not explicitly stated this appears to be the state that Cotard's syndrome patients are held to be in. As this specific issue then has such a bearing on how we deal with Metzinger's analysis we should take some time to investigate it here.

A first, minor, point is the contradiction that I noted earlier between the statements made by Metzinger (i) that there is always a transparent part of the self model and that (ii) the complete\footnote{I don't mean complete here in the sense that it is used in the psychological literature regarding Cotard's patients but rather in the sense of a patient that has reached the furthest point in the syndrome and is outright denying the existence of their self. As noted earlier Cotard's is a continuum syndrome and all comments in this section are taken to address the most nihilistic end of the spectrum.} Cotard's syndrome patient has a fully opaque self- model (i.e. there is no transparent part of the model). Resolving this contradiction is a tough task. Metzinger specifically describes the existence of a transparent part of the self-model (which I have argued is equivalent to the pre-reflexive self) as being "invariant" which would seem to preclude describing subjects with divergent (from the norm, e.g. Cotard's syndrome patients) phenomenological experiences as being exceptions or, with some cause, ruled out of the classification. On the other hand, the statements in the section specifically concerning Cotard's syndrome and the above statement that seems to concern Cotard's are very clear that there are phenomenal states where all of the content of the self-model is opaque. There is no clear way to resolve this contradiction\footnote{Particularly as Metzinger's argument as to why we should give credence to this possibility is not sufficiently strong to overturn the earlier invariant rule. While we can concede that intuitive implausibility does not rule out metaphysical possibility there is some danger of circularity in assuming that mineness and experiential centredness can be drawn apart and using this to explain Cotard's when Cotard's appears to be the best example for making this assumption (although we could potentially see some religious states as being a good example also).} but, in consideration of the subject matter of the current paper, it seems that we should probably make the judgement that applies most directly to the Cotardian case (i.e. the second of the two options given above).

The second concern to be raised relates to the notion of an entirely opaque self-model in the context of Cotard's syndrome. The Cotardian condition seems to, for Metzinger, consist in seeing through the illusory representation of the self, leading to a "view from nowhere" \cite[p. 88]{grush2000} that nevertheless remains centred on the body. However, we have also seen that the proposed cause of this situation is the loss of bioregulatory feedback in conscious experience. As we discussed, in Husserlian terms he believes that the leib, the sense of the body as it is lived, disappears from the phenomenal landscape. It seems strange then that, in the analysis of Cotard's syndrome in his own terms, the undisturbed element of phenomenal experience is the one that involves relating the body to the environment. There are various other conditions in which our conscious bodily awareness and self-location are interfered with such as Out of Body Experiences (OBE) and perceived disembodiment. In both of the cited examples, the subject experiences their point of view as shifted away from the body. Of course, Metzinger might make the point that these conditions are entirely separate from Cotard's syndrome and should not be considered when we discuss the syndrome. However, perceived disembodiment is frequently a result of acute depression (a symptom Metzinger believes to be key in generating Cotard's syndrome) and OBEs are frequently caused by damage to the Temporo-Parietal junction \cite{blanke2005}, which is also an area to which damage has been observed in Cotard's syndrome patients \cite{enoch1991,young1995}. In support of Metzinger this evidence does seem to suggest that we can successfully disassociate the self from the body that it should be centred upon as the origin of its point of view, but it does not seem to suggest that the self can be successfully disassociated from the possession of a point of view in toto\footnote{In both of the cited syndromes the subject still has a point of view even if it is no longer centred on the body that is generating that subject.}.Of course, it remains a live possibility that Cotard's may be the disorder that proves that such a disassociation is possible but a conclusion of that sort will rely on empirical data (it may be hard to get such data, I will return to this point in a later part of the paper). Presently lacking such empirical data, it remains hard to see this as a likely conclusion while we maintain an insistence that the processes leading to ongoing bioregulatory feedback and bodily awareness are impaired in the Cotard's syndrome patient.

Leaving aside these points, what is the consequence of having a truly opaque self-model? What are the hallmarks of opaque content? As I have demonstrated the content that is held, in the case of Cotard's syndrome, to be absent is equivalent to what Sartre describes as pre-reflexive self-awareness (or in Metzinger's terms "the phenomenal quality of ‘prereflexive self- intimacy" \cite[p. 459]{metzinger2003}) and more specifically it is the sense of being a subject or the sense of ownership of one's perceptions (internal and external) that is missing in this case. Making use of Metzinger's concept of the PMIR that I introduced towards the beginning of this paper we should view this situation as involving, rather than a subject-object relation, an object-object relation\footnote{In the course of the representation of the self becoming opaque it becomes directly accessible for reflexive thought and thus an object.}. It is not the case that the PMIR disappears completely from conscious experience; the body perseveres as the centre of phenomenal experience so there must still be a relation of the sort described by the PMIR\footnote{Of course, I have raised doubts about this section of Metzinger's analysis but I am setting these doubts aside for the purpose of exploring fully the consequences of following it through. It should also be noted that the persistence of the PMIR (which subsumes some of the functionality of the leib) is a case where Metzinger's explicit account diverges somewhat from the phenomenal interpretation I have made of his use of the k\"{o}rper/leib distinction.}. The originating element\footnote{For clarity I will use the following rule for referring to the two poles in the PMIR in this section: The pole that in non-deviant phenomenal states is the subject is referred to as the originating element, the other pole (the object) is referred to as the targeted element.} of the PMIR would seem to come under conscious control, in effect the Cotard's syndrome patient would become aware of the fact that they had previously been "looking through a state in [their] head" \cite[p. 335]{metzinger2003} at the world and that this condition no longer pertained. Rather than an experience of "directly" relating (by merely attending) to the targeted element of the PMIR, the patient must now become aware of the representational nature of the entity doing the relating as well as the relationship between the originating element (the centre of perceptual experience, what was once "me") and the targeted element. They experience first-order intentionality in the same way as the non-divergent subject experiences a second-order intentional experience. It is interesting to consider the question of whether this self- representation is open to endogenous control as the opaque self- representations experienced by non-divergent subjects are. I can pose a range of what-if questions\footnote{I can think about my self as it was some time ago, I can think about how things would be different if I changed an aspect of my self (the self and the person are poorly differentiated in thoughts of this kind). Even the non-divergent subject can probably not feel these thoughts at the level of the pre-reflexive self, but of course the Cotard's syndrome patient would be unable to feel anything on this level anyway.} about my self and I can place myself in the self posited by those what-if questions. Thought experiments of this sort do not take place as a subjective, internal process \cite[p. 585]{metzinger2003} but are a view as from outside, and it may be that this is a unique way in which the Cotard's syndrome patient can interact with intentional experiences that would be characterised by maximal immediacy in the non-divergent subject and thus not open to endogenous control of this sort. Situated within the present the Cotard's syndrome patient may well be able to create a conscious, objective model of their bodily states and their emotional responses although it appears from the symptomotology of the disorder that this is one of the key skills that they lack.
