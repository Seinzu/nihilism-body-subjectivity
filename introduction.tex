%-----------------------------------------------------
% Chapter: Introduction
%-----------------------------------------------------

\chapter{Introduction}
\label{chap:intro}

There are a variety of perspectives from which we might try to investigate conscious experience and the role of subjectivity within this experience. We can approach it from the phenomenological perspective, which concentrates on an account of the experiential level (although to characterise phenomenology as pure introspection would be unfair). We can look on the neurobiological level for the neural correlates of consciousness, both in general and for specific acts of consciousness (e.g. colour consciousness). We can direct our attention to the functional level by creating computer models of whole subsystems within the brain/mind complex. All of these approaches have their strengths and weaknesses. Thomas Metzinger has developed a framework and set of constraints by use of which he hopes to be able to explain the nature of conscious experience. In keeping with the foregoing, Metzinger’s discussion of consciousness and subjectivity respects the multiple methodologies we may use to investigate consciousness. Amongst other things he is seeking “the point at which objective, third-person approaches to the human mind can be integrated with first-person, subjective, and purely theoretical approaches” \cite{metzinger2003}.

One of the tools that Metzinger uses is the study of cases that diverge from standard phenomenal experience. Investigations of this sort can help us to determine the effect that changing certain phenomenal properties can have on the overarching experience had by a subject.These investigations can also, in Metzinger's words, "reveal...implicit assumptions, helps to dissolve intuitive fallacies and make...conceptual deficits of existing theories clearly visible" \cite[p. 213]{metzinger2003}. It is with Metzinger's analysis of one such divergent phenomenological experience, Cotard's syndrome, that this paper is concerned.

Cotard's syndrome is a mental disorder first described by Jules Cotard\footnote{There is, however, mention of a patient described by Charles Bonnet in the 18\textsuperscript{th} Century with many of the symptoms of Cotard's syndrome\cite{forstl1992}.}, a French Doctor, in the late 19\textsuperscript{th} Century. Enoch and Trethowan describe nihilistic delusions as being the "central essential symptom of Cotard's syndrome" \cite[p. 172]{enoch1991}. Generally, the syndrome is presented as involving the patient making claims regarding their death or in the more severe cases that they do not exist at all. I will present some more detailed remarks regarding this syndrome shortly.

Metzinger's main thesis is that "no such things as selves exist in the world" \cite[p. 1]{metzinger2003} and the experience of having a self "emerges if a conscious information processing system operates under a transparent self-model"\footnote{I will return to Metzinger’s thesis in more detail in the first main section of this paper and will explain the workings of the framework he has constructed more clearly at that point.}. As such it should be easy to see that Cotard's syndrome with its nihilistic delusions, including delusions about the patient's self, is an important measure of Metzinger's framework. It is also a subject of great interest on its own merit, regardless of any impact findings from study of the syndrome may have on Metzinger's framework.
This paper will concern itself primarily with a discussion of Metzinger's analysis of the target syndrome and an evaluation of how well this analysis stands up to evidence from the empirical literature. As such the paper will begin with a review of the empirical literature, moving on to review Metzinger's analysis of Cotard's syndrome before concluding with a comparison of his analysis and the empirical literature. I will conclude the paper by discussing ways in which Metzinger's explication of Cotard's syndrome can be improved upon.

