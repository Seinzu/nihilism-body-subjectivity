\chapter{Metzinger's analysis in light of the psychological literature}
\label{chap:psych_lit}

\section{Partitioning of theoretical space in the psychological explanation of Cotard's syndrome}
\label{psych_partitioning}

Psychological explanations of Cotard's syndrome tend to be based on one (or both) of two chief causal powers:
\begin{enumerate}
\item An affective component
\item An intellectual component\footnote{I have taken the wording for this from Berrios and Luque's \cite{berrios1995b} description of a theory held by Tissot \cite{tissot1921delire}.}
\end{enumerate}

An explanation in terms of the first causal power, Gerrans \cite{gerrans2003} and Metzinger's \cite{metzinger2003} theories for example, would take the position that "delusions are rationalizations of anomalous experiences" \cite[p. 47]{gerrans2003}. The root cause of the disease is seen to be some form of lack of emotional responsiveness. The patient then using "normal" rationality draws the conclusion (from the feedback, or lack thereof, that she receives from her body) that her body is a dead (unresponsive) body. Young and Leafhead exemplify the other major position currently considered as an explanation of the syndrome (incidentally, the Tissot paper cited previously is another, much earlier, example). They argue that we must understand Cotard's syndrome in terms of both an affective deficit and a delusional intellectual response to the stimulus presented \cite[pp. 164-165]{young1995}. A third position of a straight deficit in reasoning causing Cotard's syndrome without need to bring affective flattening into the description also exists as a possible explanation. It is beyond the scope of the current paper to consider the validity of the one-phase models as opposed to the two-phase models (to use Gerrans' terminology) except insofar as it is briefly covered in the next section, however it is valuable to have an understanding of the theoretical backdrop and the location of Metzinger's position in this context.

\section{Metzinger and the psychological literature}
\label{psych_lit_metzinger}

In my exploration of Metzinger's view of Cotard's syndrome in the previous section I outlined a number of predictions both implicit and explicit42 that Metzinger makes about the Cotardian patient and I recount them here, before presenting supporting or contradictory information from the psychological literature regarding the syndrome:

\begin{enumerate}
    \item Lack of empathy.
    \item Difficulty co-ordinating social behaviour.
    \item Insensitivity to pain.
    \item Difficulty with bodily co-ordination
    \item Difficulty in maintaining a correct division of self
    and world.
    \item Loss of ability to locate oneself spatially.
    \item Severe cases involve loss of subjective representation
    of self (self-as-object).
    \item Flattened/non-existent affect.
\end{enumerate}

In section \ref{chap:framework} of this paper I described Metzinger's analysis of Cotard's syndrome in terms of two separate arguments. The presentation of the two arguments is such that we might understand them to be merely statements of the same argument in different terminologies (and this is how I have treated them). However, it is worth noting that in section \ref{metzinger_self_sartre} I showed that the second argument made by Metzinger (that which does not directly employ the k\"{o}rper/leib distinction) does differ in an important respect; the leib which in the first argument is taken to vanish in the Cotard's syndrome patient does not appear to vanish entirely, rather that element which deals with centring of phenomenal experience remains in normal function while the senses of ownership and selfhood disappear from the phenomenological landscape. As such, predictions 4, and 6 may be taken to apply only in the case of the first argument and may well not apply to the second argument\footnote{Although it seems sensible in disorders that present symptoms so divergent from normal experience to consider that the categories by which we define the structure of our conscious experience may need to be revised, I have previously detailed that the justification for the division that Metzinger makes is not conclusive so we must keep open the possibility that such division is unwarranted.}.

\subsection{Empathy and Social Behaviour}

As empathy has its main function in helping to facilitate normal, functional social behaviour I will address the first two items on the list together. Empathy, the perception\footnote{Recognition of the emotional state of others is not an intellection but a perception, Metzinger describes the process of empathy in the following way: "Looking into each other's faces we frequently directly see the emotion expressed" \cite[p. 168]{metzinger2003}. This is not to say that there is no intellectual component (we would do well to remember Merleau-Ponty's observation that "in man the emotional life is ‘shot through with intelligence'" \cite[p. 179]{merleauponty1962}) to emotional awareness regarding the other (for instance, emotional states recognised without the tools of vision or well concealed emotions betrayed by a careless phrase) but that the primary mode relied upon is direct, transparent and perceptual.} of the affective states of others, is in Husserlian phenomenology (as well as the work of Merleau-Ponty) a process in which we feel the other within our own self or to put it in language friendly to Metzinger's analysis, we perform a simulation of that organism with which we are emphasising (by putting our own self or leib into the state of the other).

Bearing in mind the comparison made of Metzinger and Sartre in section \ref{metzinger_self_sartre} of this paper, this functionality must be taken to inhere in that part of the PSM that becomes opaque during the course of Cotard's syndrome. As such we should expect to observe impairments to the empathic capabilities of the Cotard's syndrome patient. It is not clear that such impairments would by necessity transfer over to difficulties in social interaction. It is possible that instead of pre-reflectively acting towards other people the Cotardian patient is able to replicate normal action by
using an explicit higher-order representational method to generate appropriate behavioural strategies. Phrased in these terms it is possible to see a similarity between Cotard's syndrome and autism\footnote{Hirstein makes an analysis of Cotard's syndrome in terms of the patient suffering from a complete inability to represent minds (both their own and the other's) \cite[p. 127]{hirstein2005} and this has some similarity to Baron-Cohen's description of the cause of autism as lacking a theory of mind \cite{baron1985autistic}. Obviously the terms in which we have been speaking are somewhat removed from the theory-theory espoused by Baron-Cohen but there seems to be a similarity between Zahavi's \cite[P. 218]{zahavi2005} analysis of Autism and this analysis of Cotardian social behaviour.} at least in terms of social behaviour. We might expect to detect some difficulties with complex social situations (it is not clear that Cotard's syndrome patients would have difficulties with the Sally-Anne test, they may have advanced enough representational methods to pass the test) and some levels of discomfort in social situations, although expression or even experience of such discomfort may be blocked by the lack of affective response in the patient.

There is little systematic work on the social behaviour in the literature. Enoch and Trethowan note "a double orientation" \cite[p. 159]{enoch1991} which enables sufferers to maintain behaviour (both socially and in terms of general world oriented tasks) which fits within bounds of normalcy while maintaining a complex delusion that would seem to contradict such behaviours. A majority of the case reports suggest troublesome behaviour socially but it is unclear whether these behaviours are to be explained in terms of a failure of this double orientation or as a simple failure of social behavioural skills as described in my previous discussion. There is a little more evidence regarding the effects Cotard's syndrome has on empathic abilities. All of the three cases reported by Young and Leafhead were tested for their ability to recognise facial expressions (amongst other facial processing tasks) and two of the three were found to have below normal ability to perform expression recognition. Naturally, we cannot consider this a large enough group to speak without fear of mistake, but it certainly seems a promising sign of a defect in the empathic ability of the Cotardian patient.

\subsection{Pain}

Pain is not commonly held to be an emotion \cite[p. 71]{damasio2000} and thus the absence of it does not come in to the phenomenology of the Cotard's syndrome patient through the affective flattening element of Metzinger's analysis. However, as bioregulatory feedback goes experiencing pain must be seen to have an urgent utility.

While establishing that pain is not an emotion, Damasio argues that the sensation of pain and the emotional responses to pain have been successfully disassociated experimentally so it is correct to say that pain is not an emotion, but there are emotional states that are reliably caused as a result of the same physical, body state changes that cause the pain \cite[p. 75]{damasio2000}. It is clear that affective flattening should remove the emotional component of pain (suffering etc.) but it remains an open question whether the sensational aspect of pain should also be absent. The pre- reflexive self, which makes up the invariantly transparent component of the PSM is generated by and sensitive to those elements of bioregulatory feedback that are closest to equilibrium across the maximum range of body states. Pain states are (in the normal case) divergent from the standard body state and as such pain might be considered not to constitute an essential part of the pre-reflexive self\footnote{If I might be permitted to return to Sartre's discussion of pre-reflective consciousness we can find therein a description of pain as being pre-reflectively given and thus a part of the pre-reflexive self. In the case that one becomes so consumed in other action that the pain being felt is forgotten he observes that "if I happen to gain knowledge of it [the pain] in a later reflective act it will be given as having always been there" \cite[p. 437]{sartre1956}. The questions of whether pains and emotions are part of the pre-reflexive self are most likely the same question (in the sense that to answer one in all probability means the same answer is given for the other). On the one hand, they seem to be states that apply to the self, it would be very rare for a person to identify with their pain rather than to view the pain as something that is happening to them, on the other such conceptualisation of the self is reflective thought, we are considering ourself and the pain and relating them above the level of the pre-reflective. The most likely answer seems to be that both pains and emotions can become part of the self for periods of time – we must remember that the pre-reflective self may give the appearance of being invariant (leading to the postulation of the transcendent ego by the layman) but is in fact a frequently changing, loose allegiance of sensations.}. However, the same is true of many (if not all) emotional states and a pain state can apply to many of the elements of the body schema that do constitute the pre- reflexive self. Pain is an element of bioregulatory feedback and must qualify as part of the "logic of survival" which Metzinger claims is "suspended" \cite[p. 459]{metzinger2003} for the Cotard's syndrome patient. Although it seems that a case can be made that neither of these types of pain should be present in the Cotardian patient we can divide the pain question into two separate questions\footnote{I believe, and have argued in the previous footnote, that these two questions will have identical answers on Metzinger's analysis of Cotard's syndrome, and bearing in mind his theory of self, but they can in principle be separated thus they are presented in that way here.}:

\begin{enumerate}
    \item Does the Cotardian patient experience the emotional components of pain?
    \item Does the Cotardian patient experience the direct sensation of pain?
\end{enumerate}

Damasio makes it clear that most of the external manifestations of the experience of pain might be as readily attributed to unconscious as to conscious processes so we are unfortunately reduced to patient self- reports to provide evidence regarding these questions. The literature shows little evidence for anaesthesia as a major symptom of Cotard's syndrome; it does appear in Enoch and Trethowan's review of much of the currently extant (as of 1992) literature as an "accessory symptom" \cite[p. 174]{enoch1991}. Neither, Young and Leafhead's review of Dr. Cotard's original patients nor Berrios and Luque's statistical analysis of 100 patients make any mention of analgesia. Young and Leafhead do include anaesthesia as one of the potential perceptual abnormalities in Dr. Cotard's cases but there are no cases noted which showed such a symptom. They report one patient who stabbed himself with a knife to prove the absence of blood in his body but there is no mention of whether either form of pain consciousness (see the two questions above) was experienced by the patient. It is unclear whether the lack of information on analgesic or anaesthetic symptoms in Cotard's syndrome patients is because such symptoms are rare or because physicians do not investigate such symptoms, but in either case it does appear that it is not a major, definitional symptom of Cotard's syndrome.

To digress from pain a little, there is some mention of other forms of bodily perception in the literature that may well help our consideration of the pain questions. Young and Leafhead report the case of JK who while claiming explicitly that she had died was prepared to assent to being able to feel her heart beat and awareness of several other bodily sensations \cite[p. 158]{young1995}. From the account given it isn't clear which of the two types of feeling given previously are assented to by the patient in the dialogue described. As a result of not being able to narrow down the possibilities there are a number of possible explanations available here.

Firstly, is the possibility that the sensation still passes through to conscious experience but all of the emotional resonance is not similarly available, in essence although awareness of the body is still possible the use of it meaningfully (in a pre-rational way) becomes impossible.

Secondly, it is possible that both the sensations and their connected emotions are getting through to conscious experience. This option, of course, poses deep difficulties for Metzinger's analysis of Cotard's syndrome. If the transparent component of the PSM is defined by bioregulatory feedback and the syndrome is caused by the loss of transparency in this component of the model (caused by the loss of bioregulatory feedback) then a situation in which the bioregulatory feedback is not lost but the symptoms supposedly caused by the opacity of the entire PSM are still occurring would doom his entire analysis\footnote{There is a potential escape from this line of argument in the continuous nature of the syndrome. It is possible that bioregulatory feedback disappears from phenomenal experience not as a whole but in smaller chunks with the specific content reported by the patient in this case being amongst the last to go. The fact that Young and Leafhead recognise the specific case as being the most typically Cotardian of the cases that they present and that the case appears quite complete in having already reached the stage of asserting death means that taking this escape route will leave a lot of difficult issues to resolve.}. As such the first option is the only one that Metzinger can support. Taking the first option involves denying that bodily sensations (which will include pain) help to constitute the transparent component of the PSM or that they are amongst the weaker elements of the PSM and are the last to vanish from the Cotard's syndrome patient's phenomenal landscape. Even taking this option poses some difficulties for the one-stage theories (of which, as previously discussed, Metzinger's is one) as the sort of claim that one can be dead while still having bodily sensation suggests some sort of delusional state. The patient is no longer reacting rationally to bizarre bodily feedback (i.e. no feedback) but is instead ignoring some valid feedback to come to their incorrect judgement. This wouldn't necessarily be a problem in the case of a continuum disorder such as Cotard's, as one could say that the feedback that does get through is becoming progressively less rich and that a full case of Cotard's syndrome would demonstrate a complete lack of bioregulatory feedback. Even in this case it remains troublesome that this patient has reached such an advanced stage of the syndrome (at the point of declaring death), which should preclude any clear bodily awareness.

Although the empirical literature is sadly sparse on information similar to that provided in the case of JK, both regarding pain sensations and bodily sensations in general, the information that we do have access to suggests that this is a problematic area for Metzinger's analysis of the syndrome. Either of the two options he might take to explain these symptoms are problematic and even taking the lesser of the two evils still leaves his analysis less convincing than a richer analysis that allows some role for some delusional content in the Cotardian patient.

\subsection{Bodily coordination}

As observed at the end of section \ref{metzinger_emotional_disembodiment} Metzinger is aware that there is little or no effect upon bodily co-ordination evident in the Cotard's syndrome patient. As Metzinger does not explicitly acknowledge the connection between his use of körper and leib and Husserl's (this connection is covered in section \ref{metzinger_emotional_disembodiment} and \ref{metzinger_self_husserl}) it is not clear whether he is aware of the disparity between the two modes of expression of his analysis of Cotard's syndrome. In any case, as stated there is little evidence of problems with bodily co-ordination for the Cotard's syndrome patient but this only necessarily counts against the phenomenological presentation of Metzinger's analysis. There is no need for further discussion of this point.

\subsection{Self and world}

Regarding the expected difficulty in recognising the division between self and world, there appears to be some evidence that such difficulties do occur in practice. Enoch and Trethowan note some cases in which "patients may be convinced of a massive increase in the size of their bodies, believing that they may even extend to merge with the universe" \cite[p. 174]{enoch1991}. In some cases we even find "patients [who were] at one time protesting that they did not exist...claim to be all-pervading, filling the earth" \cite[p. 174]{enoch1991}. It would be hard to think of a clearer example of an inability to make a self-world differentiation.

This is a consequence that seems to be ill covered by Metzinger's preferred explanation for Cotard's syndrome. Metzinger's analysis of the total loss of mineness does seem to imply the breakdown of the self-world boundary as a necessary consequence \cite[p. 307]{metzinger2003}. Yet in his explicit account he maintains centredness and perspectivalness \cite[p. 460]{metzinger2003} oriented around the body, which suggests the maintenance of a self-world border. We should also note that even though the self-model is greatly altered in the Cotard's syndrome patient there is still an active (but opaque) self-model which also implies a self-world boundary\footnote{Sections \ref{metzinger_self_husserl} and \ref{metzinger_self_sartre} explore this in more detail.}.

\subsection{Self-location}

Successful self-location is at the centre of Metinger's second argument; the subject loses contact with his bioregulatory feedback and yet retains his ability to centre his phenomenal experience on his body. It seems that in the case of Metzinger's first argument where the leib disappears wholesale from the Cotard's syndrome patient's phenomenal landscape.

Problems with self-location would lead to symptoms such as those found in Alien Hand syndrome on one side (mistaking a body part for that of another) and on the other a belief that one's self is located in the body of another. Alternately we can understand Out of Body Experiences (OBE) and depersonalisation to be examples of the effects of loss of spatial connection between the body and the self of this sort. There is no evidence that any symptoms of this sort exist in patients with Cotard's syndrome\footnote{Although Alien Hand isn't observed in Cotard's patients, a number of patients do claim that limbs (or other body parts) are missing when they are in fact still present (e.g. Enoch and Trethowan's Case 5 \cite[p. 172]{enoch1991} or Young and Leafhead's KH \cite[p. 160]{young1995}. However, claims of this sort relate less to body location and more to the trend of nihilistic belief generally.}.

\subsection{Subjective representation of self}

The severe form of Cotard's syndrome as I have identified its representation in Metzinger's analysis is characterised by nihilistic delusions regarding the patient's existence. According to Berrios and Luque \cite{berrios1995b} this is the third most common symptom observed in the 100 cases that they reviewed (occurring in 69\%) and this appears to lend credence to Metzinger's assertions regarding the patient experiencing a loss of sense of self. However, there is little discussion by Berrios and Luque as to the criteria by which one can be considered to be explicitly nihilistic. The assumption that all are like Capgras and Daumezon's \cite{capras1936} Madame Zero may not be justified. For example, in Young and Leafhead's table summarising the original 8 patients of Dr Cotard they classify ~60\% of the patients as negating the self, including the whole body \cite[p. 152]{young1995} which may just involve a denial of bodily existence rather than a denial of the perspectival self entirely. However, one such case suffices for Metzinger's analysis so debate along these lines need not be further pursued.

\subsection{Affect}

Most importantly, and key to Metzinger's analysis is that patients should lack affective responsiveness. This part of his analysis is similar to the suggestion made by Blakeslee and Ramachandran that "in Cotard's...all the sensory areas are disconnected from the limbic system, leading to a complete lack of emotional contact with the world" \cite[p. 167]{blakeslee1998}.

Further than this, Metzinger believes that the Cotardian patient does not have access to her own internal state. The internal "logic of survival" of the patient, which is normally communicated emotionally, is no longer phenomenally accessible causing the central nihilistic delusions that mark the syndrome. Interestingly, Damasio notes that "emotions automatically provide organisms with survival-oriented behaviour" \cite[p. 56]{damasio2000}. Metzinger follows Damasio quite closely on this topic (Damasio is the origin of "logic of survival" for example \cite[p. 198]{metzinger2003}) so we would expect this loss of emotion to not only cause nihilistic delusions but also to cause the loss of automatic survival-oriented behaviour. I have considered the possibility of the adoption of a conceptually mediated schema for handling social situations and it may well be that a similar technique must be employed to co-ordinate survival-oriented behaviour.

Metzinger seems to consider the presence of depression\footnote{He \cite[pp. 457-458]{metzinger2003} cites Gerrans \cite[p. 112]{gerrans2000} as making precisely this connection.} to indicate, or cause, lack of affective response. We should therefore also expect that more severe depression Berrios and Luque identify that depression occurs in 89\% of cases which seems to provide support for Metzinger, however they identified that patients with symptoms constituting a pure form of Cotard's syndrome "showed no loadings for depression...and included most of the complete cases" and further that "the authors reviewed here tend to consider that the completeness of the Cotard syndrome is not a specific function of the presence or severity of depression" \cite[p. 187]{berrios1995b}.

There is much discussion on this topic, certainly more than can be covered in the space available here, however we should consider that depression might well be a symptom of the syndrome rather than the chief cause. Furthermore, depression is not present in all cases and some of these cases include the most severe. However, for the sake of argument I will allow that depression is present and has a strong causal role in the disease. Trethowan and Sims note that "lowering of mood [in depression]...may be altogether absent" and that "flattening of affect" is most often observed in schizophrenic patients \cite[p. 91]{trethowan1983}. Depression is frequently accompanied by anxiety or negative mood rather than completely flat emotional response. Enoch and Trethowan present a case report which contains the following: "she did not seem to be particularly depressed but, if anything, gave the impression of being somewhat affectively flattened" \cite[p. 170]{enoch1991} in which the clear implication is that depression is not often accompanied by flattened affect. It seems that it is not necessarily the case that the presence of depression alone even means that the affective flattening needed for Metzinger's analysis to go through is present. However, this need not be altogether problematic for Metzinger, Berrios and Luque \cite[p. 187]{berrios1995b}, Ahlheid \cite[p. 927]{ahlheid1968} and several other authors cited in Enoch and Trethowan \cite[pp. 177-178]{enoch1991} note that there is a strong possibility of an organic psychosis or general delusional psychopathology being a key causal element in (at least some cases of) the syndrome. It might still be found that affective flattening is present in all or the majority of cases\footnote{Although it may be troubling for the theory of an affective causal basis for the syndrome that Enoch and Trethowan describe hallucinations that are an occasional symptom of the syndrome "are more likely to be illusions based on strong affect" \cite[p. 174]{enoch1991} than hallucination proper, suggesting that there is at least a subset of Cotardian patients who have affective states (see also my earlier discussion of pain for further discussion of a related topic).} of Cotard's syndrome but it is not clear that the depression and any emotional flattening observed is not a response to such a shocking and unpleasant experience as believing oneself to be dead or even non-existent. Another approach to the same point is taken by Young and Leafhead when they note that "it is...clear that abnormal feelings...cannot in themselves be a sufficient cause of the Cotard delusion...[because] a lack of emotional responsiveness...[is] experienced by many people who do not then draw the conclusion that they have died" \cite[pp. 164-165]{young1995}.

\section{Alternative explanations of Cotard's syndrome}
\label{psych_lit_alternatives}

The analysis presented in this section so far suggests that there are a number of problems with the analysis that Metzinger has made of Cotard's syndrome. Evidence from Berrios and Luque suggests that affective flattening does not seem to be connected directly with the severity of the syndrome, nor (according to reports from Young and Leafhead) does a complete loss of bioregulatory feedback accompany even quite severe cases. Aside from these practical concerns it is difficult to find a significant justification for the idea that perspectival centredness and pre-reflexive self-awareness can be effectively split apart when both seem (on Metzinger's analysis) to depend on bodily feedback. A successful defence of his analysis of Cotard's syndrome might provide the justification required but for the reasons given above it is difficult to award such status to Metzinger's analysis. However, there are many positive contributions to be found, both in Metzinger's analysis and in his framework as a whole. I will consider an alternative explanation that will use the PMIR and the PSM from Metzinger's framework along with a pair of explanations in entirely different terms in the remainder of this section.

\subsection{Alternative conception of subjective experience}

The central problem with Metzinger's analysis is the tension between his conception of subjective experience (specifically pre-reflective self- awareness) as being the by-product of consciously presented bodily feedback and the fact that this feedback does not appear to vanish for the Cotardian patient when he predicts that it will. It is possible that the problem does not lie directly with his analysis of Cotard's syndrome but with this conception of subjectivity. If bodily sensation does not have so important a role in the generation of pre-reflective self-awareness then the disappearance of such awareness will not necessarily imply the absence of sensational content. Of course, some adjustments to the case made by Metzinger must be made, for instance we would need to find a cause for the loss of a major element of subjective experience to replace affective flattening.

I will not consider this in detail because to settle this issue (the validity of a given way of explaining subjectivity) would require a much more space than currently available. A brief note on the difficulty facing this explanation is in order however; even if we allow that the analysis given by Metzinger of the severe cases (those in which the element of subjective experience in question is totally missing) we still need to find an explanation of the milder cases in which patients may report being dead concurrently with reporting sensations such as a beating heart. Although it helps us avoid the tension in Metzinger's analysis, adopting an alternate explanation of subjectivity does not really supply us with much greater explanatory power than taking Metzinger's. In spite of this it remains a live possibility and is worth further consideration.

\subsection{Defect in conscious ownership of feelings}
\label{psych_alternatives_defect}

During the exploration of Metzinger's conceptualisation of self in section \ref{metzinger_self} it was mentioned that the key agent in bundling self-related phenomenal content is a feeling of mineness which we could also express as ownership. This sense of ownership can operate on several different levels, Metzinger's analysis of Cotard's syndrome turning on one of these levels of ownership, ownership at a pre-reflective level. There is also ownership on the level described by Metzinger as introspection4 \cite[pp. 36, 302]{metzinger2003}\footnote{On page 302 of Being No One this type of introspection is referred to as "the level of self- directed cognition".} or "consciously experienced cognitive self-reference". This sense of ownership falls outside of the scope of the invariably transparent self- related content but as I shall speak of it in this section will in fact always be experienced as transparent.

One of the statements commonly made by Cotard's syndrome patients is that the reason they believe themselves dead is on account of not having "proper feelings" (4). Metzinger makes one interpretation of this when he takes the position that the Cotard's syndrome patient lacks a pre-reflective sensation of mineness about their experiences. We could also view that the lack of "proper" feelings may reflect some inability to reconcile the feelings that the patient is experiencing with their reflective sense of ownership. Rather than the feelings no longer being presented to them at all, they are presented as normal but some higher level process fails to operate in order to tie the ownership of these feelings with the feelings themselves creating a sense of alienation from the feelings which leads to them being classified as "not proper". This would explain why patients might still experience sensations and emotions while they report themselves dead. This also allows more comfortably for the variety of different nihilistic delusions that patients exhibit (I will touch on this later in the course of section \ref{psych_alternatives_confab}) as they are rationalisations of this more complicated phenomenal situation\footnote{If the feelings are presented to the patient as absent in the way that Metzinger describes then there are fewer potential conclusions that the patient can draw (they are forced towards loss of self and bodily death). Metzinger believes that the Cotardian patient is making a rational response to unusual stimulus. On the current account there seems to be a degree of irrationality in the patient's response to the raw stimulus provided which allows for more possible responses, corresponding to different irrationalities.}.

There are similarities between this way of treating Cotard's syndrome and the way in which Metzinger explains schizophrenia. Metzinger explains the experience of schizophrenia as involving being "confronted with conscious, cognitive contents for which they have no sense of agency or ownership" \cite[p. 445]{metzinger2003}. However, this explanation of Cotard's syndrome is not to be understood as merely being schizophrenic delusion about a specific cognitive content, the PSM\footnote{Although schizophrenia is present in many cases \cite[p. 185]{berrios1995b}[p.175]{enoch1991} it is by no means present in all or even a majority (see the Enoch and Trethowan reference above).}, but rather a different, possibly related, type of delusion. The clearest difference being that the Cotard's syndrome does not seem to experience loss of agency over the PSM, instead only losing ownership.

The big challenge to this way of explaining Cotard's syndrome is how it can deal with the severe cases, specifically those in which self-existence is totally denied. A related problem is encountered by Metzinger's analysis\footnote{That the source texts for Enoch and Trethowan's citing of these extreme cases of self- nihilism makes it difficult to say exactly what statements are made by affected patients.}, even in the case of Madame Zero, the fact that the patient has a name or an indexical of reference for herself creates a difficult situation. Saying:
\begin{enumerate}
    \item I am person x\footnote{Or even something like "this is person x".}
    \item Person x does not exist
\end{enumerate}
Begs the question of what the patient believes herself to be referencing in statement 1. Viewing this as merely a limit of the language that the patient is using to express herself doesn't avoid this problem. If this limitation is the case then the true meaning of the patient, the phenomenal experience being referred to by the patient, cannot be deduced by an observer so we have no means of differentiating between a group of interpretations of their statement. One such interpretation is the one that Metzinger makes but the current hypothesis is certainly another. The confusion of having subjective experience but failing to understand it as owned could easily cause the patient to begin to doubt the existence of their self. If nothing is happening to you then your reflective objectification of your self will become alienated from the subjective experience, cease making sense as a division of the world and thus cease to exist \cite[p. 313]{metzinger2003}. We should expect total loss of any form of self- awareness to have similar belief forming abilities as for the form lost in Metzinger's analysis. If we want to draw a conclusion as to which of these forms of self-awareness is actually operant in the Cotard's syndrome patient then we need to look at the origin of that form of self-awareness to divine the expected results of losing it and investigate whether these results are consistent with the symptoms of the Cotardian patient. It is clear from previous discussion (specifically sections \ref{chap:framework} and \ref{psych_lit_metzinger}) that the form of self-awareness lost in Metzinger's analysis is not a likely culprit. The current account can still support the continuous nature of Cotard's syndrome, as we assume that the loss of ownership from the PMIR is a gradual process, resulting at the far end in total loss of self-world boundary (leading to the belief of being all-encompassing or perhaps of losing the location of the self within the world, hence believing it ceased existence).

To place this explanation in terms of Metzinger's PMIR, there are two points at which it could occur:
\begin{enumerate}
    \item As reflective thought regarding the self is inherently objectifying and it is this type of thought that is impaired it could be that the defect is in the target element of the PMIR.
    \item The problem appears to be related to the process of intentionality so the defect might lie in the intentional arrow.
\end{enumerate}
Phenomenally these two defects should be indistinguishable but they can be distinguished on a functional level and, in all probability, on a neurobiological level.

In the first case, the object component of the PMIR during self-directed thought contains much of the content of the intransigently transparent component of the PSM but this representation lacks the information that the subject owns the content in question. Functionally this would localise the deficit, we shouldn't notice similar defects in other forms of content and we should expect the neurobiological deficits behind the syndrome to be localised to the, potentially distributed, minimal neuronal profile of the representations concerned.

In the second case, the intentional arrow must provide the phenomenal experience of ownership. The content of the reflective self-model is still intact as expected, but attention cannot be turned to it while maintaining a sense that it is "my self-model". We would expect this sort of deficit to affect a wider range of content\footnote{Although, because mineness only applies to self related content we should still only see the defect in such content but it should relate to all such content whereas, dependant on architecture, we might expect gaps in the affected content in the first case.} but it would still likely be neurobiologically localised to the neural mechanism that causes the higher-order binding which provides the sense of mineness to intentional targets.
Distinguishing which of these two options is the most likely explanation will rely on a more thorough understanding of how the experience of self is actually generated within the brain.

\subsection{The Confabulatory Explanation}
\label{psych_alternatives_confab}

In section \ref{chap:cotards} of this paper, I explored the possibility of an element of dishonesty or confabulation in the Cotard's syndrome patient, concluding that there was some evidence that suggested it might play a role in the explanation of the syndrome.

Do cases like the one cited earlier in which the cause of death was clearly confabulated mean that we can expect to find that all of the Cotardian symptoms are in fact a result of a process of confabulation? There are some indications that this might be the case. Most confabulatory episodes seem to have at least some extent of wish-fulfilment; for example in the case of Korsakoff's syndrome the confabulatory patient is acting out what might be seen as a fantasy of a fully working memory. As has been mentioned previously, and, in fact, is central to Metzinger's analysis, depression has a high incidence in Cotard's sufferers, suicide ideation has also been observed to have a high incidence and many of the cases cited by Enoch and Trethowan \cite{enoch1991} have an element of self-blame, including taking responsibility for the illnesses of other patients (pp. 163,168) or a feeling of being "completely worthless" (pp. 170). The Cotard's syndrome patient might just be insulating herself from the reality that she is not dead as she had so strongly hoped, either to escape a life that is characterised by suffering or because they feel death is deserved as a result of imagined crimes they have committed.

The problem facing confabulatory accounts is that there is little effect on our analytical treatment of the disorder as the patient may still have the underlying feelings of possessing a dead body or even lacking a self entirely (or any one of a range of nihilistic delusions). A confabulatory patient is not only trying to convince their interlocutors but also herself and she seems to have a greater degree of success with the latter. Therefore we might expect them to have the same experience as the non- confabulatory Cotard's syndrome patient is believed to have.

Regardless of the phenomenal similarity, the confabulatory case might preclude adoption of Metzinger's analysis of Cotard's syndrome. Confabulation belongs on the reflective level of self-awareness; confabulatory patients are not pre-reflexively aware of the desires that cause the confabulation, such desires are a reflective autobiographical type of thought \cite[p. 313]{metzinger2003}. If this is the case then the feeling of loss of self is more likely to be of the reflective type described in section \ref{psych_alternatives_defect} rather than the pre-reflexive type in Metzinger's analysis.

There are problems of practicality facing the use of confabulation as an explanatory tool in the Cotardian case. When dealing with confabulatory instances in, for example, Korsakoff's, it is possible to test the veracity of the patient's claims. We know for a fact that a did not go to the conference yesterday because she was in the hospital at the time. For private, subjective experience it isn't, presently, possible to discover whether a statement made by a patient matches up to the reality of their experience. At present it doesn't seem like a hypothesis that can be tested, but it remains a possible explanation of the syndrome and we can hope that improvements in neuropsychiatry may render such testing possible.

\section{Empirical remarks}

A theme that has persisted through the preceding discussion of the empirical literature has been that there is a lack of systematically gathered empirical information on many of the criteria that I derived from Metzinger's discussion of Cotard's syndrome. This shouldn't come as too much of a surprise, first Cotard's syndrome is a rare disorder so the availability of subjects for systematic research is simply not there. Secondly, with a disorder as serious as Cotard's the first priority must be alleviation of the symptoms which leads to the third reason, improved treatment has frequently stopped the progress of the syndrome before it gets to highly advanced stages (2). This could also be the reason why all the patients that may have demonstrated the total loss of subjective experience all appear in papers over fifty years old.

This scarcity of empirical data is particularly unfortunate considering the vital importance that the syndrome might have for helping refine our explanations of the phenomenal self. Many of the criteria provided at the beginning of this section could be tested informally simply by gathering the information required by direct questioning or observation of patient behaviour. 

There is some potential for the use of Trans-cranial Magnetic Stimulation (TMS)\footnote{Due to the time requirements imposed by the administration of capability experiments (in order to find deficits in simulated Cotard's syndrome subjects), repetitive TMS (rTMS) would seem to be preferable to the use of a single pulse \cite[p. 297]{weiner2003handbook}, use of rTMS will increase the risk of side-effects slightly.} in simulating Cotard's syndrome. There are indications in the literature that damage to the temporo-parietal lobe \cite{enoch1991}{young1995} may be a cause of Cotard's syndrome. Other conditions that involve a degree of alienation from the body (specifically OBE) may result from damage to a near area, specifically the temporo-parietal junction \cite{blanke2005}, suggesting that this area may be involved in the phenomenal experience of body image. Could application of TMS to this area cause a reaction much like Cotard's syndrome? It is unclear whether this would work, as Enoch and Trethowan note that although there is a connection between the temporo- parietal lobe "a disorder of behaviour is more likely to be the result of generalized [sic] cerebral involvement" \cite[p. 177]{enoch1991} in addition to some lesions in the tempo-parietal lobe\footnote{Incidentally this seems to also motivate against Metzinger's analysis of Cotard's syndrome, if a lesion to the part of the brain most clearly involved in maintenance of the body schema is not sufficient a causing the syndrome then there would seem to need to be a pre-existing defect in reasoning powers in addition to the altered bodily feedback. As such performing appropriate rTMS on the temporo-parietal junction might well be a good test of Metzinger's analysis. For the reasons given in previous parts of this paper it seems  unlikely to be sufficient so it might not be helpful in narrowing the field of alternate explanations.}. Some experiments with rTMS have revealed an effect to the emotional responses of subjects \cite{wasserman2000}, which might cause some problems if there is an affective element to the syndrome.

It seems, at present, that it is unlikely that a good TMS based simulation of Cotard's syndrome can be made but it should be possible to perform some experiments based on causing temporary lesions (with TMS) to the area of the brain implicated in management of the body schema which may well help to refine our understanding of how the self and body interact and will bring us closer to an understanding of Cotard's syndrome. Additionally, as I have noted in a previous footnote, we can use rTMS to test the accuracy of Metzinger's analysis of the syndrome.

On the whole there are many challenges facing attempts to gather sufficient empirical data to formulate a thorough explanation of Cotard's syndrome, scarcity of patients, efficacy of treatments preventing the development of the syndrome into the more interesting severe form and difficulties with current lesion simulation technology. Of these problems, only the last is likely to be alleviated in the future, we can expect the number of naturally occurring severe cases to reduce (which is naturally a positive development, generally speaking).