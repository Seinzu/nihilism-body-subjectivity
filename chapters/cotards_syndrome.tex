\chapter{Cotard's Syndrome}
\label{chap:cotards}

As has been mentioned previously, Jules Cotard reported Cotard’s syndrome in the late 19
\textsuperscript{th} Century under the term d\"{e}lire des negations, the name Cotard’s syndrome being provided later by Seglas. Berrios and Luque\cite{berrios1995} note that the standard translation of this into English, nihilistic delusion, ``only manage[s] to convey fragments of [the original] French meaning" and that we should understand the condition as being more complex than being caused by a simple incorrect belief\footnote{Metzinger makes a similar observation when he says that ``classic belief-desire psychology and a traditional philosophical analysis in terms of propositional attitudes \ldots may not be helpful" \cite[p. 458]{metzinger2003}. As his specific analysis is explored it will become clearer why he would probably agree with Berrios and Luque’s position on this issue.}. Instead ``\ldots it is more like a syndrome that may include symptoms from the intellectual, emotional or volitional spheres". There is some controversy in the literature on this point\footnote{Young and Leafhead \cite{young1995} notably take the position that the nihilistic delusions present in most cases of Cotard’s are in fact merely a symptom that can appear in a variety of distinct disorders. There is some support for this position in the range of circumstances in which the delusions appear (see Enoch and Trethowan’s discussion of this for more details \cite{enoch1991}).} and, as Berrios and Luque note, this has been the case since shortly after Cotard’s death, this paper will not directly address this controversy but will instead assume that Berrios and Luque are correct in their analysis\footnote{It is not clear that the syndrome controversy will affect the validity of Metzinger's analysis of the nihilistic delusion but he takes the disorder to be a syndrome throughout his analysis and it seems to be the best fit of the two sides. If we view the nihilistic delusions present in Cotard’s to be a simple, symptomatic delusion with no common underlying affective component then there will be a clash between Metzinger’s analysis and the outcome of the syndrome controversy, but this is certainly not the only possible non-syndrome outcome. Young and Leafhead take up a non-syndrome position that is compatible on the whole with the analysis that Metzinger makes, differing only on whether the judgements made by Cotardian patients can be considered a rational response to the underlying affective component or not (they argue that it cannot).}.

\section{Symptoms}

Cotard's patients may present with both the presence and the severity of symptoms along a continuum. Young and Leafhead \cite{young1995} note that the disorder can range from \index{self!negation}``self-deprecation and feelings of despair" to ``a total denial of the self and external world" and point out that Cotard ``viewed as essential the presence of negativism and accusatory depressive delusions". I will return to the influence of \index{depression}depression on the course of the disorder when I discuss Metzinger's analysis of the syndrome towards the end of the paper. In general patients display a progression from the former to the latter (although cases where patients condition moves back and forth have been reported) with the progress of subjective negation starting ``with the denial of one specific part of the body" which proceeds to cases where the patient ``den[ies] her very existence, even dispensing altogether with the use of the personal pronoun" \cite{young1995}.

\section{Aetiology}

Dr Cotard's original belief regarding the syndrome was that it was ``a new type of depression" \cite[p. 185]{berrios1995b}, Young and Leafhead (amongst others, including Metzinger) also believe that the root of Cotard's syndrome is in the depressive illnesses. Enoch and Trethowan\cite[p. 173]{enoch1991} agree that ``\ldots this nihilistic delusion is often associated with a depressive illness" but following Ahlheid \cite{ahlheid1968} point out that there is frequently an organic component to nihilistic delusions and that it can also be expressed during the course of schizophrenia. This is further backed up by case studies such as that of one of Bonnet's patients presented by Forstl and Beats in which the patient suffers from the delusion of being dead (amongst others) seemingly as the result of a stroke \cite{forstl1992}. \index{depression} Depression is however frequently present, even amongst the organic cases (I shall return to discussion of this during section \ref{chap:psych_lit} of this paper).

\section{Honesty, confabulation and the Cotard's syndrome patient}

An important concern that one might have about the reports given by sufferers of d\"{e}lire des negations is whether the nihilistic belief\footnote{Use of the term ``belief" is justified here even if Cotard's syndrome can't eventually be explained using belief-desire psychology (meaning that such usage is not a pre-judgement of the issue relating to delusion vs. rational response to change in affective context) because statements are made by patients in the form of a belief and they attest to holding the nihilistic beliefs that are attributed to them.} in question are sincerely held by the patient. After all most people will recall occasions when, perhaps after a night of overindulgence or contracting a strain of Influenza, they have made a statement along the lines of ``I feel dead"\footnote{Or, alternately, there is the wonderful idiom ``I feel like death warmed up".} in response to enquiries from solicitous co-workers. Might it not be the case that the Cotard's syndrome sufferer is merely making an analogy between their condition and death? Alternately we might consider the statements to be a peculiar expression of a deeply held desire \cite{turnbull2004}. The patient, for whatever reason, wishes that they were dead (or wishes they were missing an arm, whatever happens to be the target of their nihilistic belief) and makes a statement as if the desired state had already been attained. For clarity's sake, we are postulating an element of bad faith in the statements made by the patient. They may be immune to correction in their statements about their condition but they are aware that the statements are not true reflections of their experience. Obviously it is difficult to prove either way what a person truly feels when they make the nihilistic statements, but there is some evidence from the psychological literature that may lend some support to the integrity of the Cotardian patient.

Firstly, Cotard's syndrome patients do not create coherent stories that are optimal for convincing sceptical interrogators. Young and Leafhead describe the case of JK who ``claimed that she was dead" \cite[p. 157]{young1995} but when asked ``whether she could feel her heart beat\ldots[or] feel hot or cold" she nevertheless answered that ``she could" \cite[p. 157]{young1995}. Assuming the normal belief that neither of these things is possible in the case of a dead person, this would be an odd allowance to make if there was intent to deceive. It would be very simple to merely deny having such feelings and if pushed by proof, for example the doctor setting up a heart monitor, to deny the veracity of the method used. Young and Leafhead report that the patient was aware of the disparity between her description of her ``dead" state and the understanding generally held regarding the state of dead people, which suggests that her beliefs regarding death included the common conception (with an additional possible concept that described her current state). Cotard's syndrome patients frequently emphasise that it would not be possible for others to understand their current condition, which suggests an awareness of the standard conception of the state of death. To phrase it otherwise, they are able to read the other's mind, which may be a key element in successfully orchestrating a falsehood \cite[p. 101]{hirstein2005} and yet they are unable to put the falsehood into action successfully which suggests that they lack intent to deceive, so they probably sincerely hold the belief. Another of Young and Leafhead's patients provides us with further evidence that the belief is held sincerely. KH, a \index{depression}depressive patient, ``claimed that he had no blood in him" \cite[p. 160]{young1995} to back this statement up he cut open his arm with a knife in demonstration of his bloodlessness. This case would also seem to rule out the possibility that the Cotardian patient is talking metaphorically about her condition.

There is another potential source of unreliability in the Cotard's syndrome patient in the form of \index{confabulation}confabulation. Confabulation differs from lying or simple dishonesty in the absence of intent to deceive. Hirstein \cite[p. 16]{hirstein2005} defines lying in the following way:

\begin{displayquote}
I lie to you when (and only when) \begin{enumerate}
    \item I claim p to you.
    \item p is false
    \item I believe p is false
    \item I intend to cause you to believe p is true by claiming that p is true.
\end{enumerate}
\end{displayquote}

Other than the third, all of these requirements are still present in the confabulatory case. Of the examples that I gave at the beginning of this section the most apt is the wish fulfilment hypothesis. The desire is held strongly enough that the patient, in order to avoid severe disappointment, is forced to believe it literally to be true and to convince herself of the veracity of her story. There are other potential reasons that the confabulator might concoct their stories, the most common in other cases of confabulation seems to be to cover an unpleasant deficit. For example, confabulation frequently occurs in memory loss (notably \index{Korsakoff's syndrome}Korsakoff's syndrome), with patients providing false memories in response to questions about what they had done the previous day.

Confabulation does appear to be present in Cotard's syndrome patients at least regarding the autobiographical stories that they tell regarding their condition\footnote{Enoch and Trethowan also note confabulation as a possibility in the general case \cite[p. 177]{enoch1991}.}. Young and Leafhead tell of one case (citing Young and colleagues \cite{young1992}) of a man (referred to as WI) who developed Cotard's syndrome symptoms following a head trauma. At various points during the time he was suffering from nihilistic delusions he concocted stories to explain his death: ``\ldots he was convinced \ldots that he had died of septicaemia \ldots or perhaps from AIDS". The former cause of death being a potential complication of the accident causing the head trauma and the latter apparently being the result of a news report concerning a person with AIDS dying from septicaemia (which seems to be an embellishment of the first story).

I will return to the possibility of confabulation in the Cotard's syndrome patient during the course of section \ref{chap:psych_lit} of this paper.